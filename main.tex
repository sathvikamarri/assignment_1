\documentclass[a4paper,12pt,twocolumn]{article}
\usepackage[utf8]{inputenc}
\usepackage{amsmath}
\usepackage{amssymb}
\usepackage{multicol}
\usepackage{graphicx}
\begin{document}

{\title{Assignment I (ICSE-2019 CLASS 10)}}
{\author{Sathvika marri - AI21BTECH11020}}

\maketitle
\textbf{{ICSE-2019 CLASS 10}}\\\\
\textbf{Question: 6 (a)}\\\\
(a) In the given figure, \angle PQR = \angle PST = 90$^{\circ}$, PQ = 5 cm and PS = 2 cm.\\\\
(i) Prove that \triangle PQR \hspace{0.1cm} \sim \hspace{0.1cm}  \triangle PST.\\\\
(ii) Find \hspace{0.1cm} ratio \hspace{0.1cm} of \hspace{0.1cm} Area \hspace{0.1cm} of \triangle PQR \hspace{0.1cm} and\hspace{0.1cm}\\
Area \hspace{0.1cm}  of \hspace{0.1cm} quadrilateral \hspace{0.1cm} SRQT.\\\\
\includegraphics[width=6cm]{triangle PQR.png}\\\\
\textbf{Solution:- }\\\\
(i)\hspace{0.1cm}  To \hspace{0.2cm} prove \hspace{0.2cm} \triangle PQR \hspace{0.1cm} \sim \hspace{0.1cm} \triangle PST\\\\
consider \hspace{0.1cm}  \triangle PQR \hspace{0.2cm} and \hspace{0.2cm} \triangle PST\\\\
\angle PQR = \angle PST \hspace{0.1cm} = 90$^{\circ}$ \hspace{0.1cm} (given)\\\\
\angle p \hspace{0.1cm} is \hspace{0.1cm} common\\\\
\therefore \hspace{0.1cm} \triangle \hspace{0.1cm} \sim \hspace{0.1cm} \triangle PST \hspace{0.2cm} (By \hspace{0.1cm} AA \hspace{0.1cm} criterion)\\\\

(ii)To find the ratio of area of \triangle PQR \hspace{0.1cm} and \hspace{0.1cm} area \hspace{0.1cm} of \hspace{0.1cm} quadrilateral \hspace{0.1cm} SQRT.\\

Now,
\[\frac{Ar \triangle PQR}{Ar \triangle PST}\]\\\\
=$\frac{{\[\frac{1}{2}\]}\times PQ \times QR}$/$\frac{{\[\frac{1}{2}\]}\times PQ \times QR}$\\\\
=$\frac{5}{2} \times \frac{5}{2}$ = $\frac{25}{4}$ \hspace{0.1cm}[\because \frac{PQ}{PS}=\frac{QR}{ST}]\\\\
Taking \hspace{0.1cm} the \hspace{0.1cm} reciprocals\hspace{0.1cm} on \hspace{0.1cm}both \hspace{0.1cm} sides\\\\
\[\frac{Ar \triangle PST}{Ar \triangle PQR}\] = $\frac{4}{25}$\\\\
Now deducting both sides by 1,\\\\
1-$\frac{Ar \triangle PST}{Ar \triangle PQR}$= 1-$\frac{4}{25}$\\\\
\implies 
\[\frac{Ar \triangle PQR - AR \triangle PST}{Ar \triangle PQR}\] \hspace{0.1cm}=\hspace{0.1cm} $\frac{25-4}{5}$\\\\
\implies
\[\frac{Ar of quadrilateral SRQT}{Ar of \triangle PQR}\]=$\frac{21}{25}$

THE END

\end{document}

